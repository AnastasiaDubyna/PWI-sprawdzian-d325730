\documentclass[12pt]{article}
\usepackage[utf8]{inputenc}
\usepackage[polish]{babel}
\usepackage{courierten}
\usepackage[T1]{fontenc}
\usepackage[a4paper, total={6in, 9in}]{geometry}
\usepackage{relsize}
\usepackage{amssymb}
\usepackage{amsmath}
\usepackage{graphicx}
\usepackage{hyperref}
\graphicspath{{.}}

\title{\vspace{-2.0 cm}Sprawozdanie v.2}
\author{Anastasiia Dubyna }
\date{November 2020}

\begin{document}

\maketitle

\section* {Zadanie I}
\begin{itemize}
    \large\item $\rho \frac{D \mathbf{u}}{Dt} = \rho\bigg(\frac{\delta \mathbf{u}}{\delta t} + \mathbf{u} + \nabla \mathbf{u}\bigg) = \\ = -\nabla \bar{p} + \nabla \cdot \bigg\{\mu \bigg(\nabla\mathbf{u} + \big(\nabla\mathbf{u} \big)^T - \frac{2}{3} \big(\nabla \cdot \mathbf{u} \big) \mathbf{I} \bigg)\bigg\} + \rho\mathbf{g} $
    \large\item $\tilde{f} \big(\xi\big) = \int^\infty_{-\infty} f\big(x\big) \: e^{-2\pi i x \xi} \: dx $
    \large\item $\mathbb{P}\bigg( \hat{X}_n - z_{1-\frac{\alpha}{2}} \frac{\sigma}{\sqrt{n}} \leqslant \mathbb{E}X \leqslant \hat{X}_n + z_{1-\frac{\alpha}{2}} \frac{\sigma}{\sqrt{n}} \bigg)$
    \large\item $ 
    \begin{bmatrix}
    1 & 2 \\
    3 & 4
    \end{bmatrix}
    \otimes
    \begin{bmatrix}
    0 & 5 \\
    6 & 7
    \end{bmatrix}
    =
    \begin{bmatrix}
    1 &
    \begin{bmatrix}
    0 & 5 \\
    6 & 7
    \end{bmatrix}
    & 2 &
    \begin{bmatrix}
    0 & 5 \\
    6 & 7
    \end{bmatrix} \\
    3 &
    \begin{bmatrix}
    0 & 5 \\
    6 & 7
    \end{bmatrix}
    & 4 &
    \begin{bmatrix}
    0 & 5 \\
    6 & 7
    \end{bmatrix} \\
    \end{bmatrix}
    =
    \begin{bmatrix}
    0 & 5 & 0 & 10 \\
    6 & 7 & 12 & 14 \\
    0 & 15 & 0 & 20 \\
    18 & 21 & 24 & 28
    \end{bmatrix}
    $
\end{itemize}
\newpage
\section*{Zadanie II}
\begin{enumerate}
    \item Wygenerowałam klucze za pomocą polecenia ssh-keygen \\
    \includegraphics[scale = 0.8]{1.jpg}
    \item Za pomocą ssh-copy-id skopiowałam klucz publiczny na serwer \\
    \includegraphics[scale = 0.8]{2.jpg}
    \item Stworzyłam na GitHubie repozytorium PWI-sprawdzian-d325730 i dodałam do niego drugi klucz publiczny \\
    \includegraphics[scale = 0.8]{3.jpg}
    \newpage
    \item Stworzyłam plik konfiguracyjny, który pozwala zalogować się na serwer za pomocą tylko polecenia ssh pwi-sprawdzian\\
    \includegraphics[scale = 0.8]{4.jpg}
    \item W pliku konfiguracyjnym włączyłam ForwardAgent, co pozwala na przekierowanie kluczy lokalnych tak by były dostępne również na serwerze.Tworzenie nowej pary kluczy byłoby rozwiązaniem brzydkim ponieważ wtedy każda osoba która ma dostęp do serwera również będzie miała dostęp do naszego klucza prywatnego. 
\end{enumerate}
\section*{Zadanie III}
\begin{enumerate}
    \item Skopiowałam repozytorium na serwer za pomocą git clone i pobrałam plik ze skosa za pomocą polecenia wget
    Screen nie zawiera tych poleceń, bo to wszystko zrobiłam w trakcie kolokwium i nie widzę potrzeby w ponownym kopiowaniu repozytorium ściągnięciu pliku, żeby to zademonstrować skoro zadanie jest proste
    \includegraphics[scale = 0.8]{5.jpg}
    \item Rozpakowałam archiwum za pomocą polecenia tar -xvf \\
    -x: odczytuje podane pliki z archiwum\\
    -v: wyświetla nazwy dołączanych plików \\
    -f: używa archiwum nazwę którego podaliśmy dalej\\
    \includegraphics[scale = 0.8]{6.jpg}
    \item Dodałam oraz zakomitowałam zmiany\\
    \includegraphics[scale = 0.8]{7.jpg}
    \item Niestety jest już prawie druga w nocy i nie mam sił na zrobienie zadania z archiwum. Postaram się ogarnąć czym jest funkcja skrótu jutro, ale oczywiście już nie będę mogła dodać tego do sprawozdania. 
\end{enumerate}
\section*{Bibliografia}
\begin{itemize}
    \item \url{https://www.youtube.com/watch?v=hQWRp-FdTpc}
    \item \url{https://www.overleaf.com/learn/how-to/How-to_Guides}
    \item \url{https://www.rpi.edu/dept/arc/training/latex/LaTeX_symbols.pdf}
    \item \url{https://www.linux.pl/man/index.php?command=tar}
    \item \url{http://edukacja.3bird.pl/download/informatyka/etap4/gentoo/informatyka-etap4-gentoo-sshd-conf.pdf}
\end{itemize}
\end{document}
